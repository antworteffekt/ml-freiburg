%%%%%%%%%%%%%%%%%%%%%%%%%%%%%%%%%%%%%%%%%
% Beamer Presentation
% LaTeX Template
% Version 1.0 (10/11/12)
%
% This template has been downloaded from:
% http://www.LaTeXTemplates.com
%
% License:
% CC BY-NC-SA 3.0 (http://creativecommons.org/licenses/by-nc-sa/3.0/)
%
%%%%%%%%%%%%%%%%%%%%%%%%%%%%%%%%%%%%%%%%%

%----------------------------------------------------------------------------------------
%	PACKAGES AND THEMES
%----------------------------------------------------------------------------------------

% \documentclass[10pt,handout]{beamer}
\documentclass[10pt]{beamer}

\usepackage{amsmath}
\usepackage{amssymb}
\usepackage{xcolor}
\usepackage[utf8]{inputenc}
\usepackage[T1]{fontenc}
\usepackage{textcase}

\mode<presentation> {

% The Beamer class comes with a number of default slide themes
% which change the colors and layouts of slides. Below this is a list
% of all the themes, uncomment each in turn to see what they look like.

\usetheme{progressbar}
\progressbaroptions{headline=sections}


% \usecolortheme{albatross}
\usecolortheme{beaver}
% \usecolortheme{beetle}
% \usecolortheme{crane}
% \usecolortheme{dolphin}
% \usecolortheme{dove}
% \usecolortheme{fly}
% \usecolortheme{lily}
% \usecolortheme{orchid}
% \usecolortheme{rose}
% \usecolortheme{seagull}
% \usecolortheme{seahorse}
% \usecolortheme{whale}
% \usecolortheme{wolverine}

%\setbeamertemplate{footline} % To remove the footer line in all slides uncomment this line
%\setbeamertemplate{footline}[page number] % To replace the footer line in all slides with a simple slide count uncomment this line

%\setbeamertemplate{navigation symbols}{} % To remove the navigation symbols from the bottom of all slides uncomment this line
}

\usepackage{graphicx} % Allows including images
\usepackage{booktabs} % Allows the use of \toprule, \midrule and \bottomrule in tables

\setbeamercovered{dynamic}
\definecolor{dgreen}{rgb}{0.,0.6,0.}
\newcommand{\beispiel}[1]{\textcolor{dgreen}{\textbf {#1}}}


%----------------------------------------------------------------------------------------
%	TITLE PAGE
%----------------------------------------------------------------------------------------

\title[Abalone]
{Abalone age prediction}
 % The short title appears at the bottom of every slide, the full title is only on the title page


\author{José Luis Licón, Max Lotstein} % Your name
\institute[Uni Freiburg] % Your institution as it will appear on the bottom of every slide, may be shorthand to save space
{
Albert-Ludwigs-Universität Freiburg \\ % Your institution for the title page
\medskip
\textit{emails} % Your email address
}
\date{\today} % Date, can be changed to a custom date

\begin{document}
\setbeamertemplate{caption}{\insertcaption}



\begin{frame}
\titlepage % Print the title page as the first slide
\end{frame}

\begin{frame}
\frametitle{Overview} % Table of contents slide, comment this block out to remove it
\tableofcontents % Throughout your presentation, if you choose to use \section{} and \subsection{} commands, these will automatically be printed on this slide as an overview of your presentation
\end{frame}

%----------------------------------------------------------------------------------------
%	PRESENTATION SLIDES
%----------------------------------------------------------------------------------------

%------------------------------------------------
\section{Introduction}
% Sections can be created in order to organize your presentation into discrete 
% blocks, all sections and subsections are automatically printed in the table of 
% contents as an overview of the talk
%------------------------------------------------
\begin{frame}\frametitle{Model design}
  The topology of the multi-layer perceptron was constrained as follows:
  \begin{itemize}
    \item Always 10 input nodes
    \item Always 3 output nodes (discretisation of age variable)
    \item 1 and 2 hidden layers were tested for significant differences in 
    performance
  \end{itemize}
\end{frame}
%------------------------------------------------


%------------------------------------------------
\begin{frame}\frametitle{Training parmeters}

The following training parameters were modified:
\begin{itemize}
  \item Regularisation parameter (to enforce weight decay)
  \item Optimisation method (gradient descent, GD with momentum, GD with
  acceleration, Bayesian regularisation, etc.)
  \item Network topology
\end{itemize}

\end{frame}
%------------------------------------------------

%------------------------------------------------
\begin{frame}\frametitle{Tools}
  
\begin{columns}[c]
\column{.5\textwidth}
An initial attempt was made to use Neurolab. In the interest of simplicity we
switched to MATLAB's neural network toolbox.

\column{.5\textwidth}
\begin{figure}[htbp]
  \centering
    \includegraphics[height=2.5in]{graphics/MatLab-NN-Interface.PNG}
  % \caption{caption}
  % \label{fig:graphics_MatLab-NN-Interface}
\end{figure}

\end{columns}

\end{frame}
%------------------------------------------------

%------------------------------------------------
\begin{frame} \frametitle{Exploratory data analysis}
\begin{figure}[htbp]
  \centering
    \includegraphics[scale=.22]{graphics/rings.png}
\end{figure}


\end{frame}
%------------------------------------------------


\section{Neural network results}
%------------------------------------------------
\begin{frame} \frametitle{Parameter effect}
  Effect on training error:
  \vfill
  {\scriptsize
  \begin{tabular}{lrrrrr}
  \toprule
  \multicolumn{1}{l}{}&\multicolumn{1}{c}{Df}&\multicolumn{1}{c}{Sum Sq}&\multicolumn{1}{c}{Mean Sq}&\multicolumn{1}{c}{F value}&\multicolumn{1}{c}{Pr(\textgreater F)}\tabularnewline
  \midrule
  NumHiddenLayers                           &$  1$&$ 0.00$&$0.00$&$0.01$&$0.9338$\tabularnewline
  HiddenLayerSize                           &$  2$&$ 0.01$&$0.01$&$0.10$&$0.9056$\tabularnewline
  TrainingFn                                &$  6$&$ 3.30$&$0.55$&$8.33$&$0.0000$\tabularnewline
  NumHiddenLayers:HiddenLayerSize           &$  2$&$ 0.01$&$0.00$&$0.06$&$0.9424$\tabularnewline
  NumHiddenLayers:TrainingFn                &$  6$&$ 0.00$&$0.00$&$0.00$&$1.0000$\tabularnewline
  HiddenLayerSize:TrainingFn                &$ 12$&$ 0.01$&$0.00$&$0.01$&$1.0000$\tabularnewline
  NumHiddenLayers:HiddenLayerSize:TrainingFn&$ 12$&$ 0.01$&$0.00$&$0.02$&$1.0000$\tabularnewline
  Residuals                                 &$168$&$11.08$&$0.07$&$$&$$\tabularnewline
  \bottomrule
  \end{tabular}
}
\end{frame}
%------------------------------------------------

%------------------------------------------------
\begin{frame} \frametitle{Parameter effect}
\begin{figure}[htbp]
  \centering
    \includegraphics[scale=.22]{graphics/training_error.png}
\end{figure}

\end{frame}
%------------------------------------------------

%------------------------------------------------
\begin{frame} \frametitle{Parameter effect}
  Effect on test error:
  \vfill
  {\scriptsize
  \begin{tabular}{lrrrrr}
  \toprule
  \multicolumn{1}{l}{}&\multicolumn{1}{c}{Df}&\multicolumn{1}{c}{Sum Sq}&\multicolumn{1}{c}{Mean Sq}&\multicolumn{1}{c}{F value}&\multicolumn{1}{c}{Pr(\textgreater F)}\tabularnewline
  \midrule
  NumHiddenLayers                           &$  1$&$0.04$&$0.04$&$ 4.02$&$0.0464$\tabularnewline
  HiddenLayerSize                           &$  2$&$0.02$&$0.01$&$ 0.70$&$0.4967$\tabularnewline
  TrainingFn                                &$  6$&$0.98$&$0.16$&$14.95$&$0.0000$\tabularnewline
  NumHiddenLayers:HiddenLayerSize           &$  2$&$0.01$&$0.00$&$ 0.27$&$0.7669$\tabularnewline
  NumHiddenLayers:TrainingFn                &$  6$&$0.03$&$0.01$&$ 0.48$&$0.8243$\tabularnewline
  HiddenLayerSize:TrainingFn                &$ 12$&$0.02$&$0.00$&$ 0.18$&$0.9990$\tabularnewline
  NumHiddenLayers:HiddenLayerSize:TrainingFn&$ 12$&$0.04$&$0.00$&$ 0.29$&$0.9903$\tabularnewline
  Residuals                                 &$168$&$1.83$&$0.01$&$$&$$\tabularnewline
  \bottomrule
  \end{tabular}
  }
  
\end{frame}
%------------------------------------------------

%------------------------------------------------
\begin{frame} \frametitle{Parameter effect}
\begin{figure}[htbp]
  \centering
    \includegraphics[scale=.22]{graphics/test_error.png}
\end{figure}

\end{frame}
%------------------------------------------------

%------------------------------------------------
\begin{frame} \frametitle{Confusion matrices}

  \begin{columns}[c]
  \column{.3\textwidth}
  \begin{figure}[htbp]
    \centering
      \includegraphics[scale=.1]{graphics/conf0_0.pdf}
    \caption{$\lambda = 0$}
    \label{fig:graphics_conf0.0}
  \end{figure}
  
  \begin{figure}[htbp]
    \centering
      \includegraphics[scale=.1]{graphics/conf0_1.pdf}
    \caption{$\lambda = 0.1$}
    \label{fig:graphics_conf0.1}
  \end{figure}  
  
  \column{.3\textwidth}
  
  \begin{figure}[htbp]
    \centering
      \includegraphics[scale=.1]{graphics/conf0_2.pdf}
    \caption{$\lambda = 0.2$}
    \label{fig:graphics_conf0_2}
  \end{figure}
  
  \begin{figure}[htbp]
    \centering
      \includegraphics[scale=.1]{graphics/conf0_5.pdf}
    \caption{$\lambda = 0.5$}
    \label{fig:graphics_conf0_5}
  \end{figure}
  
  \column{.3\textwidth}
  
  \begin{figure}[htbp]
    \centering
      \includegraphics[scale=.1]{graphics/conf1_0.pdf}
    \caption{$\lambda = 1$}
    \label{fig:graphics_conf1_0}
  \end{figure}
  

  \end{columns}


\end{frame}
%------------------------------------------------


%------------------------------------------------
\begin{frame} \frametitle{Confusion matrices}

  \begin{columns}[c]
  \column{.3\textwidth}
  \begin{figure}[htbp]
    \centering
      \includegraphics[scale=.1]{graphics/conftrainbr.pdf}
    \caption{$\lambda = 0$}
    \label{fig:graphics_conf0.0}
  \end{figure}
  
  \begin{figure}[htbp]
    \centering
      \includegraphics[scale=.1]{graphics/conftraingd.pdf}
    \caption{$\lambda = 0.1$}
    \label{fig:graphics_conf0.1}
  \end{figure}  
  
  \column{.3\textwidth}
  
  \begin{figure}[htbp]
    \centering
      \includegraphics[scale=.1]{graphics/conftraingda.pdf}
    \caption{$\lambda = 0.2$}
    \label{fig:graphics_conf0_2}
  \end{figure}
  
  \begin{figure}[htbp]
    \centering
      \includegraphics[scale=.1]{graphics/conftraingdm.pdf}
    \caption{$\lambda = 0.5$}
    \label{fig:graphics_conf0_5}
  \end{figure}
  
  \column{.3\textwidth}
  
  \begin{figure}[htbp]
    \centering
      \includegraphics[scale=.1]{graphics/conftraingdx.pdf}
    \caption{$\lambda = 1$}
    \label{fig:graphics_conf1_0}
  \end{figure}
  
  \begin{figure}[htbp]
    \centering
      \includegraphics[scale=.1]{graphics/conftrainrp.pdf}
    \caption{$\lambda = 1$}
    \label{fig:graphics_conf1_0}
  \end{figure}

  \end{columns}


\end{frame}
%------------------------------------------------

%------------------------------------------------
\begin{frame} \frametitle{Bayesian regularisation}

\end{frame}
%------------------------------------------------


\begin{frame}
\Huge{\centerline{The End/Questions}}
\end{frame}

%----------------------------------------------------------------------------------------

\end{document}